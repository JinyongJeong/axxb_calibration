%%%%%%%%%%%%%%%%%%%%%%%%%%% asme2ej.tex %%%%%%%%%%%%%%%%%%%%%%%%%%%%%%%
% Template for producing ASME-format journal articles using LaTeX    %
% Written by   Harry H. Cheng, Professor and Director                %
%              Integration Engineering Laboratory                    %
%              Department of Mechanical and Aeronautical Engineering %
%              University of California                              %
%              Davis, CA 95616                                       %
%              Tel: (530) 752-5020 (office)                          %
%                   (530) 752-1028 (lab)                             %
%              Fax: (530) 752-4158                                   %
%              Email: hhcheng@ucdavis.edu                            %
%              WWW:   http://iel.ucdavis.edu/people/cheng.html       %
%              May 7, 1994                                           %
% Modified: February 16, 2001 by Harry H. Cheng                      %
% Modified: January  01, 2003 by Geoffrey R. Shiflett                %
% Use at your own risk, send complaints to /dev/null                 %
%%%%%%%%%%%%%%%%%%%%%%%%%%%%%%%%%%%%%%%%%%%%%%%%%%%%%%%%%%%%%%%%%%%%%%

%%% use twocolumn and 10pt options with the asme2ej format
\documentclass[twocolumn,10pt]{asme2ej}
%%\documentclass[twocolumn,15pt]


\usepackage{epsfig} %% for loading postscript figures
\usepackage{mathtools}   % loads »amsmath«
\usepackage{amssymb,amsmath}
\usepackage{amsfonts}
\usepackage{multirow}
%\usepackage{natbib}
\usepackage{relsize}
\usepackage{graphicx}
\usepackage{color}
\usepackage{comment}
%\usepackage{algorithm2e}


\newtheorem{theorem}{Theorem}[section]
\newtheorem{corollary}{Corollary}
\newtheorem*{main}{Main Theorem}
\newtheorem{lemma}[theorem]{Lemma}
\newtheorem{proposition}{Proposition}
\newtheorem{conjecture}{Conjecture}
\newtheorem*{problem}{Problem}
%\theoremstyle{definition}
\newtheorem{definition}[theorem]{Definition}
\newtheorem{remark}{Remark}
\newtheorem*{notation}{Notation}

\DeclarePairedDelimiter{\norm}{\lVert}{\rVert}

\newcommand{\II}{\mathbb{I}}
\newcommand{\overbar}[1]{\mkern 1.5mu\overline{\mkern-1.5mu#1\mkern-1.5mu}\mkern 1.5mu}
\newcommand{\ttt}{{\bf t}}





\title{New Batch Method}

\author{Qianli Ma\thanks{Address all correspondence to this author.}\\
		\textbf{M. Kendal, Ackerman}\\ 
		\textbf{Gregory S. Chirikjian}
    \affiliation{
	Robot and Protein Kinematics Laboratory\\
	Laboratory for Computational Sensing and Robotics\\
	Department of Mechanical Engineering\\
	The Johns Hopkins University\\
	Baltimore, Maryland, 21218\\
    Email: \{mqianli1, gchirik1\}@jhu.edu
    }	
}


\begin{document}

\maketitle    

%%%%%%%%%%%%%%%%%%%%%%%%%%%%%%%%%%%%%%%%%%%%%%%%%%%%%%%%%%%%%%%%%%%%%%
\begin{abstract}
\it
Abstract to be added.
 
\end{abstract}

We begin by defining a Gaussian probability distribution on $SE(3)$ (assuming the norm $\norm{\Sigma}$ is small) as
$$ \rho(H; M, \Sigma) = \frac{1}{(2\pi)^3 |\Sigma|^{\frac{1}{2	}}} e^{-\frac{1}{2}F(M^{-1}H)}$$
where $\norm{\Sigma}$ denotes the determinant of $\Sigma$ and
$$ F(H) = [\log^{\vee}(H)]^T \Sigma^{-1} [\log^{\vee}(H)].$$

Previously, in order to determine the mean of the convolution of two PDFs, Baker-Campbell-Hausdorff formula is used given the assumption that function $f_1$ and $f_2$ are both highly focused. If $X, Y \in se(3)$, 
\begin{equation} 
log(e^X e^Y) = X + Y + \dfrac{1}{2}[X,Y] + \dfrac{1}{12}\left([X, [X,Y]] + [Y,[Y,X]]\right) + ...
\label{BCH}
\end{equation}
If $X$ and $Y$ are further constrained to be small so that $\norm{X} \ll 1$ and $\norm{Y} \ll 1$, then the first approximation of Eq.(\ref{BCH}) can be written as:
\begin{equation}
\log(e^X e^Y) = X + Y
\end{equation}
As shown in \cite{Wang08}, the mean of covariance of the convolution of two highly focused functions are: 
\begin{equation}
M_{1*2} = M_1 \, M_2 \,\,\, {\rm and} \,\,\, \Sigma_{1*2} = Ad(M_2^{-1}) \,\Sigma_{1}\, Ad^T(M_2^{-1}) + \Sigma_{2}
\label{meancovconvdef} \end{equation}
where
\begin{equation} 
Ad(H) = \left(\begin{array}{ccc}
R && \mathbb{O} \\
\widehat{{\bf x}} R && R \end{array}\right) 
\label{adjdef} \end{equation}
\section{First Order Approximation of $M$}
Though this approximation works well when the distribution of $X$ is treated as a Delta function, it fails to extend to the case where its distribution is a general PDF $f(X)$. In an alternative to the first order approximation using Baker-Campbell-Hausdorff formula, it is possible to only assume $M^{-1}H$ is small so that $\norm{M^{-1}H - \II} \ll 1$. Given the Taylor expansion of the matrix logarithm described as:
\begin{equation}
\log(\II + X) = X - \dfrac{1}{2}X^2 + \dfrac{1}{3}X^3 - ...
\end{equation}
Then it is straight forward to have:
\begin{equation}
\begin{split}
&\log(M^{-1}H) \\ 
= &\log(\II + (M^{-1}H - \II)) \\ 
= &\left(M^{-1}H - \II \right) - \left(M^{-1}H - \II \right)^2/2 + \left(M^{-1}H - \II \right)^3/3 - ...
\end{split}
\end{equation}
Given the definition of the mean $M$ of a probability density $f(H)$ as:
\begin{equation} 
\int_{SE(3)} \log(M^{-1} H) f(H) dH = \mathbb{O}  
\label{meandef} 
\end{equation}
The first order approximation of Eq.(\ref{meandef}) is:
\begin{equation}
\int_{SE(3)} (M^{-1}H - \II) f(H)dH \approx \mathbb{O}
\end{equation}

\begin{equation}
M^{-1}\int_{SE(3)} H f(H)dH \approx \II
\end{equation}
Define the first order approximation of $M$ as $\widehat{M}$:
\begin{equation}
\widehat{M} \doteq \int_{SE(3)}Hf(H)dH
\label{1st}
\end{equation}

 $$ f_A(H) = \frac{1}{n} \sum_{i=1}^{n} \delta(A_i^{-1} H) \,\,\, {\rm and} \,\,\, f_B(H) = \frac{1}{n} \sum_{i=1}^{n} \delta(B_i^{-1} H). $$
 If $f(H)$ is of the form of $f_A(H)$ given above, then
\begin{equation} \
\begin{split} &\sum_{i=1}^{n} \log(M_A^{-1} A_i) = \mathbb{O} {\rm \,\,\,\,\, and} \\
&\Sigma_A = \frac{1}{n} \sum_{i=1}^{n} \log^{\vee}(M_A^{-1} A_i) [\log^{\vee}(M_A^{-1} A_i)]^T.  \label{datameancovconvdef} \end{split}
\end{equation}
Discrete version will be:
\begin{equation}
\widehat{M_{A}} \doteq \sum_{i=1}^{n} A_{i} \left( \frac{1}{n} \sum_{j=1}^{n}  \delta(A_j^{-1} A_{i})dH \right) = \dfrac{1}{n}\sum_{i=1}^{n}A_{i}
\end{equation}
Note that $\widehat{M}$ is generally not a group element in $SE(3)$, and the corresponding $SE(3)$ version can be obtained by projecting $\widehat{M}$ into $SE(3)$ using singular value decomposition (SVD) technique.:
\begin{equation}
R_{\widehat{M}} = U \Sigma V^{T}
\label{proj}
\end{equation}
The rotation part of the projected $\widehat{M}$ (named as $\widehat{M}_{proj}$) is:
\begin{equation}
R_{\widehat{M}_{proj}} = UV^{T}
\end{equation}
More details are needed to actually project $\widehat{M}$ into $SE(3)$.

\section{Second Order Approximation of $M$}
Take the second order approximation in Eq.(\ref{meandef}):
\begin{equation}
\int_{SE(3)} \left((M^{-1}H - \II) - \dfrac{1}{2}(M^{-1}H - \II)^2 \right)f(H)dH \approx \mathbb{O}
\label{2nd1}
\end{equation}
\begin{equation}
\int_{SE(3)} \left( 2M^{-1}H - \dfrac{1}{2}HM^{-1}H - \dfrac{3}{2} \II \right)f(H)dH \approx \mathbb{O}
\label{2nd2}
\end{equation}
Multiply $M$ on both side of Eq.(\ref{2nd2}):
\begin{equation}
\int_{SE(3)} \left( 2H - \dfrac{1}{2}HM^{-1}H -\dfrac{3}{2}M \right)f(H)dH \approx \mathbb{O}
\label{2nd3}
\end{equation}
Substituting Eq.(\ref{1st}) into Eq.(\ref{2nd3}), we have:
\begin{equation}
2\widehat{M} - \dfrac{1}{2}\int_{SE(3)}HM^{-1}Hf(H)dH - \dfrac{3}{2}M \approx \mathbb{O}
\end{equation}
The 2nd order approximation of $M$ is denoted by $\overbar{M}$ defined as:
\begin{equation}
2\widehat{M} - \dfrac{1}{2}\int_{SE(3)}H\overbar{M}^{-1}Hf(H)dH - \dfrac{3}{2}\overbar{M} = \mathbb{O}
\end{equation}

\begin{equation}
2\widehat{M} - \overbar{M}\dfrac{1}{2}\int_{SE(3)}\overbar{M}^{-1}H\overbar{M}^{-1}Hf(H)dH - \dfrac{3}{2}\overbar{M} = \mathbb{O}
\label{2ndm}
\end{equation}

\begin{equation}
2\widehat{M} - \dfrac{1}{2n}\sum_{i=1}^{n}A_{i}\overbar{M}^{-1}A_{i} - \dfrac{3}{2}\overbar{M} = \mathbb{O}
\label{2ndm}
\end{equation}
The second term of Eq.(\ref{2ndm}) is very similar to the definition of the covariance of $f(H)$, and maybe  $\overbar{M}$ can be updated using the information of the covariance. Also, take a look at the cubness of variance in the first volume. 
The same technique as in Eq.(\ref{proj}) can be employed to project $\widehat{M}$ into $SE(3)$.

\section{First Order Approximation of $\Sigma$}
Given the definition of the covariance $\Sigma$ of a PDF $f(H)$ as:
\begin{equation} 
\Sigma = \int_{SE(3)} \log^{\vee}(M^{-1} H) [\log^{\vee}(M^{-1} H)]^T  f(H) dH \label{meancovdef} \end{equation}
Its first order approximation $\widehat{\Sigma}$ can be written as:
\begin{equation}
\widehat{\Sigma} \doteq \int_{SE(3)} (M^{-1} H - \II)^{\vee} [(M^{-1} H - \II)^{\vee}]^T  f(H) dH
\label{1stCov}
\end{equation}
The discrete version for $\Sigma_{A}$ will be:
\begin{equation}
\Sigma_A = \frac{1}{n} \sum_{i=1}^{n} (M_A^{-1} A_i- \II)^{\vee} [(M_A^{-1} A_i- \II)^{\vee}]^T.  
\end{equation}
By defining $Q = M^{-1}H$, Eq.(\ref{1stCov}) can be written as:
\begin{equation}
\widehat{\Sigma} \doteq \int_{SE(3)} (Q - \II)^{\vee} [(Q - \II)^{\vee}]^T  f(Q) dQ
\end{equation}
If $\norm{G - \II} \ll 1$, then $\Sigma = \widehat{\Sigma}$
\begin{equation}
(Q - \II)^{\vee} =
\left( 
\begin{matrix}
\dfrac{1}{2}(R -R^{T}) \\
\ttt
\end{matrix}
\right)
\end{equation}

\section{Second Order Approximation of $\Sigma$}

\section{First Order Approximation of $M_{1*2}$}
Given two functions, $f_1, f_2 \in (L^1 \cap L^2)(SE(3))$, the convolution is defined as:
\begin{equation}
(f_1 * f_2)(H) \doteq \int_{SE(3)} f_1(K) f_2(K^{-1} H) dK. 
\label{convdef}
\end{equation}
The corresponding mean $M_{1*2}$ will be given as:
\begin{equation}
\int_{SE(3)} \log(M_{1*2}^{-1}H) (f_1*f_2)(H)dH = \mathbb{O}
\end{equation}
If both $\Sigma_1$ and $\Sigma_2$ are very small, then $M_{1*2} \approx M_1 M_2$. Without using this approximation, take the assumption that $M^{-1}H$ is small and we will have:
\begin{equation}
\int_{SE(3)}\int_{SE(3)} \left(M_{1*2}^{-1}H - \II \right)f_{1}(K)f_{2}(K^{-1}H)dKdH \approx \mathbb{O}
\end{equation}
Define $L = K^{-1}H$,
\begin{equation}
\int_{SE(3)}\int_{SE(3)} \left(M_{1*2}^{-1}KL - \II \right)f_{1}(K)f_{2}(L)dKdL \approx \mathbb{O}
\end{equation}
By using Eq.(\ref{1st}) twice,
\begin{equation}
M_{1*2}^{-1} \widehat{M_1}\widehat{M_2} \approx \II
\end{equation}
Define $\widehat{M}_{1*2}$ as:
\begin{equation}
\widehat{M}_{1*2} = \widehat{M_1}\widehat{M_2}
\end{equation}
There are two ways to solve for $M_{1*2}^{proj}$.
\begin{equation}
M_{1*2}^{proj}=
\begin{cases}
\left(\widehat{M_1}\widehat{M_2}\right)_{proj}\\
\widehat{M_{1}}_{proj}\widehat{M_{2}}_{proj}
\end{cases}
\label{b_old}
\end{equation}


\section{Second Order Approximation of $M_{1*2}$}
For simplicity, we drop the domain of integral $SE(3)$,
\begin{equation}
\int\int\left(2M_{1*2}^{-1}H - \dfrac{1}{2}M_{1*2}^{-1}HM_{1*2}^{-1}H - \dfrac{3}{2}\II \right)f_{1}(K)f_{2}(K^{-1}H)dKdH \approx \mathbb{O}
\end{equation}
Substitute $L = K^{-1}H$,
\begin{equation}
\int\int \left(2M_{1*2}^{-1}KL - \dfrac{1}{2}M_{1*2}^{-1}KLM_{1*2}^{-1}KL - \dfrac{3}{2}\II \right)f_{1}(K)f_{2}(L)dKdL \approx \mathbb{O}
\end{equation}
Employing the definition of $\widehat{M}$,
\begin{equation}
M_{1*2}^{-1}\widehat{M_{1}}\widehat{M_{2}} \approx \dfrac{1}{4} \int\int \left(\dfrac{1}{2}M_{1*2}^{-1}KLM_{1*2}^{-1}KL \right)f_{1}(K)f_{2}(L)dKdL + \dfrac{3}{4}\II 
\end{equation}



















% Here's where you specify the bibliography style file.
% The full file name for the bibliography style file 
% used for an ASME paper is asmems4.bst.
\bibliographystyle{asmems4}

% Here's where you specify the bibliography database file.
% The full file name of the bibliography database for this
% article is asme2e.bib. The name for your database is up
% to you.
\bibliography{New_Batch}
\end{document}